\documentclass[11pt]{article}
\usepackage[utf8]{inputenc}
\usepackage{graphicx}
\usepackage{hyperref}
\usepackage[a4paper, total={6in, 8in}]{geometry}

\title{Notes on Chapter 5 - Structured Types, Mutability, and High Order Functions}
\author{Swarup Tripathy \thanks{John V Guttag}}
\date{February 2022}


\begin{document}
    \maketitle
    A curated list of important points for my reference.\\
    \begin{enumerate}
        \item Literals of type \textbf{tuples} are written by enclosing a comma separated list of elements within parenthesis.
        \item Like strings, tuples can be concatenated, indexed and sliced.
        \item Sequences and Multiple Assignments
        \begin{itemize}
            \item Executing the statement x,y=(3,4) where x will be bound to 3 and y to 4
            \item The statement a,b,c = 'xyz' will bind x to a, y to b and z to c.
        \end{itemize}
        \item Built-in-function \textbf{id}, which returns a unique integer identifier for an object.
    \end{enumerate}
\end{document}