\documentclass[11pt]{article}
\usepackage[utf8]{inputenc}
\usepackage{graphicx}
\usepackage{hyperref}
\usepackage[a4paper, total={6in, 8in}]{geometry}

\title{Notes and My learning from Project Euler problems}
\author{Swarup Tripathy \thanks{Project Euler Problems}}
\date{February 2022}


\begin{document}
    \maketitle
    A curated list of important points for my reference.\\
    \begin{enumerate}
        \item Itertools are a great way of creating complex iterators which helps in getting faster execution time and writing memory efficient code.
        \item Finding the key of the max value in a dictionary $\rightarrow$ max(dictionary, key=dictionary.get)
        \item When you wish to continously iterate over a list 
        \begin{itemize}
            \item from itertools import cycle
            \item list defined
            \item cycle(list)
            \item defining a function to tell the 'next' over cycle(list)
            \item For more info: \href{https://medium.com/@masnun/infinitely-cycling-through-a-list-in-python-ef37e9df100}{Itertools and Cycle}
        \end{itemize}
        \item Whenever dealing with dates, day and months use inbuilt function datetime for ease. Be Smart!
    \end{enumerate}
\end{document}