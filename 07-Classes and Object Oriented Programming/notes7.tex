\documentclass[11pt]{article}
\usepackage[utf8]{inputenc}
\usepackage{graphicx}
\usepackage{hyperref}
\usepackage[a4paper, total={6in, 8in}]{geometry}

\title{Notes on Chapter 8 - Classes and Object Oriented Programming}
\author{Swarup Tripathy \thanks{John V Guttag}}
\date{March 2022}


\begin{document}
    \maketitle
    A curated list of important points for my reference.\\
    \begin{enumerate}
        \item Objects are the core things that Python programs manipulate. Every object has a type that defines the kinds of things that programs can do with that object.
        \item An \textbf{Abstract Data type} is a set of objects and the operations on those objects.
        \item The two powerful mechanisms for managing the complexity of programming are 
        \begin{itemize}
            \item Decomposition $\rightarrow$ Creates the structure of the program
            \item Abstraction $\rightarrow$ Suppresses the detail
        \end{itemize}   
        \item One implements data abstractions using \textbf{classes}.     
        \item [:] is a slice syntax for every element in the array.
        \item When a function definition occurs within a class definition, the defined function is called as \textbf{method} and is associated with the class. These methods are sometimes referred to as \textbf{method attributes} of the class.
        \item Class supports 2 kinds of operations
        \begin{itemize}
            \item \textbf{Instantiation} is used to create instances of the class. 
            
            For ex., the statement s = IntSet() creates a new object of type IntSet. This object is called an Instance of IntSet.
            \item \textbf{Attribute References} use dot notations to access attributes associated with the class. For ex., s.member refers to the method member associated with the instance s of type IntSet.
        \end{itemize}
        \item Whenever a class is instantiated, a call is made to the \textunderscore\textunderscore init\textunderscore\textunderscore method defined in that class.
        \begin{verbatim}
            s=IntSet()
            s.insert(3)
            print(s.member(3))
        \end{verbatim}
        creates a new instance of IntSet, inserts the integer 3 into that IntSet, and then prints true.
        \item When data attributes are associated with a class we call them \textbf{Class variables}. When they are associated with an instance we call them \textbf{instance variables}.
    \end{enumerate}
\end{document}