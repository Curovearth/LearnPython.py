\documentclass[11pt]{article}
\usepackage[utf8]{inputenc}
\usepackage{graphicx}
\usepackage{hyperref}
\usepackage[a4paper, total={6in, 8in}]{geometry}

\title{Notes on Chapter 8 - Classes and Object Oriented Programming}
\author{Swarup Tripathy \thanks{John V Guttag}}
\date{March 2022}


\begin{document}
    \maketitle
    A curated list of important points for my reference.\\
    \begin{enumerate}
        \item Objects are the core things that Python programs manipulate. Every object has a type that defines the kinds of things that programs can do with that object.
        \item An \textbf{Abstract Data type} is a set of objects and the operations on those objects.
        \item The two powerful mechanisms for managing the complexity of programming are 
        \begin{itemize}
            \item Decomposition $\rightarrow$ Creates the structure of the program
            \item Abstraction $\rightarrow$ Suppresses the detail
        \end{itemize}   
        \item One implements data abstractions using \textbf{classes}.     
        \item [:] is a slice syntax for every element in the array.
        \item When a function definition occurs within a class definition, the defined function is called as \textbf{method} and is associated with the class. These methods are sometimes referred to as \textbf{method attributes} of the class.
        \item \textbf{OBJECTS}
        \begin{itemize}
            \item They have individuality and multiple names can be bound to the same object.
            \item Known as Aliasing in other languages.
            \item A \textit{Namespace} is a mapping from names to objects.
            \item In the expression, \textit{z.real}, \textit{real} is an attribute of the object z.
        \end{itemize}
        \item Class supports 2 kinds of operations
        \begin{itemize}
            \item \textbf{Instantiation} is used to create instances of the class. 
            
            For ex., the statement s = IntSet() creates a new object of type IntSet. This object is called an Instance of IntSet.
            \item \textbf{Attribute References} use dot notations to access attributes associated with the class. For ex., s.member refers to the method member associated with the instance s of type IntSet.
        \end{itemize}
        \item Whenever a class is instantiated, a call is made to the \textunderscore\textunderscore init\textunderscore\textunderscore method defined in that class.
        \item The \textunderscore\textunderscore init \textunderscore\textunderscore  method lets the class initialize the object's attributes and serves no other purpose.
        \begin{verbatim}
            s=IntSet()
            s.insert(3)
            print(s.member(3))
        \end{verbatim}
        creates a new instance of IntSet, inserts the integer 3 into that IntSet, and then prints true.
        \item When data attributes are associated with a class we call them \textbf{Class variables}. When they are associated with an instance we call them \textbf{instance variables}.
        \item All instances of user-defined classes are hashable, and therefore can be used as dictionary keys.
        \item What actually \textbf{Hashable} means in python? Ref: \href{https://www.geeksforgeeks.org/why-and-how-are-python-functions-hashable/}{Geeks for Geeks}
        \begin{itemize}
            \item hashable is a feature of Python objects that tells if the object has a hash value or not.
            \item If the object has a hash value then it can be used as a key for a dictionary or as an element in a set.
            \item An object is hashable if it has a hash value that does not change during its entire lifetime.
            \item Python has a built-in hash method ( \textunderscore\textunderscore hash\textunderscore\textunderscore() ) that can be compared to other objects.
            \item if the hashable objects are equal then they have the same hash value.
            \item All immutable built-in objects in Python are hashable like tuples while the mutable containers like lists and dictionaries are not hashable. An example below
            \begin{verbatim}
        t1 = (1, 5, 6)
        t2 = (1, 5, 6)
        # show the id of object
        print(id(t1))
        print(id(t2))
        ######## output ########
        1954294958784
        1954294958784
            \end{verbatim}
        \end{itemize}
        \item Abstract Data Types
        \begin{itemize}
            \item  An abstract data type is a type with associated operations, but whose representation is hidden.
            \item They lead to a different way of thinking about organising large programs.
            \item Data Objects $\rightarrow$ A data object is a region of storage that contains a value or group of values.
            \item writing expressions that directly access instance variables is considered poor form and should be avoided.
        \end{itemize}
        \item \textbf{Inheritance}
        \begin{itemize}
            \item It provides a convenient mechanism for building groups of related abstractions.
        \end{itemize}
        \item You can use *args and **kwargs as arguments of a function when you are unsure about the number of arguments to pass in the functions.
        \item \textbf{ **kwargs(Keyword Arguments)}
        \begin{itemize}
            \item allows us to pass a variable number of keyword arguments to a python function. In the function, we use the double asterisk(**) before the parameter name to denote this type of argument.
        \end{itemize}
        \item \textbf{rindex}: String method finds the last occurrence of the specified value. Example given below
        \begin{verbatim}
        name = "Swarup Tripathy"
        index = name.rindex(' ')    # index = 6
        index2 = name.rindex('a')   # index2 = 11
        \end{verbatim}
    \end{enumerate}
\end{document}