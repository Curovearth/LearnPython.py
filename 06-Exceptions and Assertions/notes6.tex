\documentclass[11pt]{article}
\usepackage[utf8]{inputenc}
\usepackage{graphicx}
\usepackage{hyperref}
\usepackage[a4paper, total={6in, 8in}]{geometry}

\title{Notes on Chapter 7 - Exceptions and Assertions}
\author{Swarup Tripathy \thanks{John V Guttag}}
\date{February 2022}


\begin{document}
    \maketitle
    A curated list of important points for my reference.\\
    \begin{enumerate}
        \item an exception is usually defined as 'something that does not conform to the norm' and therefore somewhat rare. There is nothing rare about exceptions in Python. They are everywhere.
        \item Most commonly ocurring types of exceptions are 
        \begin{itemize}
            \item TypeError
            \item IndexError
            \item NameError
            \item ValueError
        \end{itemize}
        \item Given the following code 
        \begin{verbatim}
            def readVal(valType, requestMsg, errorMsg):
            while True:
                val = input(requestMsg+' ')
                try:
                    return valType(val)
                except ValueError:
                    print(val,errorMsg)
        
            readVal(int, 'enter an integer:','is not an integer')
        \end{verbatim}
        The function readVal is \textbf{Polymorphic}, i.e., it works for arguments of many different types. Such functions are easy to write in Python, since types are First Class Objects.

        What are \textbf{First Class Objects?} \href{https://stackoverflow.com/questions/245192/what-are-first-class-objects}{First Class Objects in Python}
        
        A first class object is an entity that can be dynamically created, destroyed, passed to a function, returned as a value, and have all the rights as other variables in the programming language have. 
        Depending on the language this can imply

        \begin{itemize}
            \item being expressible as an anonymous literal value
            \item being storable in variables
            \item being storable in data structures
            \item having an intrinsic identity (independent of any given name)
            \item being comparable for equality with other entities
            \item being passable as a parameter to a procedure/function
            \item being returnable as the result of a procedure/function
            \item being constructible at runtime
            \item being printable
            \item being readable
            \item being transmissible among distributed processes
            \item being storable outside running processes
        \end{itemize}
    \end{enumerate}
\end{document}