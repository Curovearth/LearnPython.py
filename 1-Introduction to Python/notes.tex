\documentclass[12pt, letterpaper, twoside]{article}
\usepackage[utf8]{inputenc}
\title{Notes for the Introductory Chapter}
\author{Swarup Tripathy \thanks{John V Guttag}}
\date{January 2022}

\begin{document}
\maketitle
A curated key points with regards to the introductory chapter. 
\begin{enumerate}
    \item As for memory, a small computer might have hundreds of gigabytes of storage. How big is that? If a byte (the number of bits, typically eight, required to represent one character)
    weighed one gram(which it doesn't), 100 gigabytes would weigh 10000 metric tons. For comparison, that's roughly the combined weight of 15000 African elephants.
    \item Heron of Alexandria was the first to document a way to compute the square root of a number.
    \item An algorithm is a finite list of instructions that describe a \textbf{computation} that when executed on a set of inputs will proceed through a set of well-defined states and 
    eventually produce an output.
    \item The first truly modern computer was the Manchester Mark 1
    \item Most often, the interpreter simply goes to the next instruction in the sequence, but not always. In some cases, it performs a test, and on the basis of that test, execution may jump
    to some other point in the sequence of instructions. This is called \textbf{flow of control}.
    \item In 1936, the british mathematician Alan Turing described a hypothetical computing device that has come to be called a Universal Turing Machine.
    \item Anything that can be programmed in one programming language (e.g., Python) can be programmed in any other programming language  (e.g., Java)
    \item Python is not optimal for programs that have high reliability constraints (because of its weak static semantic checking) or that are built and maintained by many people or over a long 
    period of time (again because of the weak static semantic checking)
    \item Python was introduced by Guido Von Rossum in 1990
    \item Within the computer, values of type float are stored in the computer as \textbf{floating point numbers.}
    \item In python, variable names can contain uppercase and lowercase letters, digits(but they can not start with a digit), and the special character.
    \item Unicode standard is a character coding system designed to support the digital processing and display of the written texts of all languages. The standard contaims more than 120,000 different
    characters - covering 129 modern and historic scripts and multiple symbol sets.
    \item You can tell python which encoding to use by inserting a comment of the form # -*- coding: encoding name -*-
\end{enumerate}

\end{document}